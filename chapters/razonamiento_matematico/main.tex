\section{Razonamiento matemático}
\subsection{Números reales}

Los números reales son aquellos que pueden representarse de forma decimal, incluyendo tanto números racionales como irracionales. Matemáticamente, el conjunto de los números reales se denota con la letra
$\mathbb{R}$.

\subsubsection{Subconjuntos de los Números Reales}

\begin{enumerate}
	\item Números Naturales ($\mathbb{N}$): Son los números enteros positivos: 1, 2, 3, 4.
	\item Números Enteros ($\mathbb{Z}$): Incluyen los números naturales, sus opuestos (negativos) y el cero: -3, -2, -1, 0, 1, 2, 3.
	\item Números Racionales ($\mathbb{Q}$): Son números que pueden expresarse como el cociente de dos enteros, donde el denominador no es cero. Ejemplos: $\frac{1}{2}, -3, 4.5$
	\item Números Irracionales: Son números que no pueden expresarse como una fracción de dos enteros. Su representación decimal es infinita y no periódica.EJemplos: $\sqrt{2}, \pi, e$.
\end{enumerate}

\subsubsection{Propiedades}
\begin{enumerate}
	\item Orden: Los números reales pueden ordenarse en una recta numérica. Cada número real tiene una posición única en esta recta.
	\item Densidad: Entre dos números reales cualesquiera siempre hay otro número real. Esto significa que no hay "huecos" en la recta de los números reales.
	\item Compleción: Los números reales completan el conjunto de los números racionales en el sentido de que toda secuencia de Cauchy (una secuencia donde los elementos se acercan indefinidamente entre sí) tiene un límite en los números reales.
	\item Operaciones: Los números reales son cerrados bajo las operaciones de suma, resta, multiplicación y división (excepto la división por cero).
\end{enumerate}

\subsubsection{Suma y resta}
\subsubsection*{Suma}
Es una operación que tiene por objeto reunir dos o mas expresiones algebraicas (sumados) en una sola expresión algebraica (suma).
Asi la suma de $a$ y $b$ es denotada por $a+b$, porque esta ultima expresión es la reunion de de las dos expresiones anteriores.
\\
\textbf{Ejemplos:}
$
	\\
	1 + 2 = 3\\
	3 + 4 = 7\\
	-1 + 2 = 1\\
	-1 + 2 + 3 = 4
$
\subsubsection*{Resta}

La resta o sustracción es una operación que tiene por objeto, dada una suma de dos sumandos (minuendo) y uno de ellos (sustraendo), hallar el otro sumando (resta o diferencia).

Es evidente, de esta definición, que la suma del sustraendo y la diferencia tiene que ser el minuendo.

Si de $a$ (minuendo) queremos restar $b$ (sustraendo), la diferencia $a - b$. En efecto: $a - b$ sera la diferencia si sumada con el sustraendo $b$ reproduce el minuendo $a$, y en efecto: $a - b + b = a$.
\\
\textbf{Ejemplos:}
\\
$
	1 - 1 = 0\\
	2 - 1 = 1\\
	-1 - 2 = -3\\
	-1 - 2 - 3 = -6
$

\subsubsection{Multiplicación y división}

\subsubsection*{Multiplicación}

La multiplicación es una operación que tiene por objeto, dadas dos cantidades llamadas multiplicando y multiplicador, hallar una tercera cantidad, llamada producto, que sea respecto del multiplicando, en valor absoluto y signo, lo que el multiplicador es respecto de la unidad positiva.

El multiplicando y multiplicador son llamados factores del producto.

El orden de los factores no altera el producto. Esta propiedad, demostrada en Aritmética, se cumple también en Algebra.Asi, el producto $ab$ puede escribirse $ba$; el producto $abc$ puede escribirse $bca$ o $acb$.

Esta es la Ley Conmutativa de la multiplicación.

\textbf{Ejemplos:}

$
	1 * 3 = 3\\
	2 * 3 = 6\\
	-1 * 3 = -3\\
	(10)(20) = 200\\
	20(3) = 60\\
$

\subsubsection*{Division}

La division es una operación que tiene por objeto, dado el producto de dos factores (dividendo) y uno de los factores (divisor), hallar el otro factor (cociente).

De esta definición se deduce que el cociente multiplicado por el divisor reproduce el dividendo.

Asi la operación de dividir $6a^2$ entre $3a$. Esa cantidad (cociente) es 2a.

Es evidente que $6a^2 / 2a = \frac{6a^2}{2a} = 3a$ donde  vemos que si el dividendo se divide entre el cociente nos da de cociente lo que antes era divisor.

\subsubsection{Ley de los signos y exponentes}

\subsubsection*{Ley de los signos}

\begin{itemize}
	\item $-ab \div -a = \frac{-ab}{-a} = +b$ porque $(-a)(+b) = -ab$
	\item $+ab \div -a = \frac{+ab}{-a} = -b$ porque $(-a)(-b) = ab$
	\item $-ab \div +a = \frac{-ab}{+a} = -b$ porque $(+a)(-b) = -ab$
\end{itemize}

En resumen:
\begin{center}
	$+$ entre $+$ da $+$\\
	$-$ entre $-$ da $-$\\
	$+$ entre $-$ da $-$\\
	$-$ entre $+$ da $-$\\
\end{center}

\subsubsection{Ley de los exponentes}
Sea el cociente $a^5 \div a^3$. Decimos que
\begin{center}
	$a^5 \div a^3 = \frac{a^5}{a^3} = a^2$\\
\end{center}
$a^2$ sera el cociente de esta division si multiplicada por el divisor $a^5$ reproduce el dividendo, y en efecto: $a^2 \times a^5 = a^5$.
\subsubsection{Raíces y potencias con exponente racional}

Las raíces y exponentes racionales son términos equivalentes, ya que ambos se refieren a la misma forma de potenciación de los términos.

Los exponentes racionales son aquellos que son expresados como el cociente de dos enteros, donde el denominador no es cero.

Ejemplos:

\begin{center}
$\sqrt{a} = a^{\frac{1}{2}}$\\
$\sqrt[3]{a^5} = a^{\frac{5}{3}}$
\end{center}


\subsection{Expresiones algebraicas}

\subsubsection{Términos}

Los términos son aquellos que componen una expresión algebraica. Un conjunto de términos es un conjunto de expresiones algebraicas. Por ejemplo:

\begin{itemize}
	\item $a^2$ es un termino.
	\item $ab^2$ es un termino.
	\item $a^2b^2$ es un termino.
\end{itemize}

\subsubsection{Polinomios}
Es la suma de uno o mas términos algebraicos cuyas variables tienen exponentes enteros positivos.
Un polinomio es un conjunto de términos en los cuales se puede pueden clasificar en 4 tipos:

\begin{itemize}
	\item Monomio: Es un polinomio que consta de solo un termino, por ejemplo: $-2x^2y^3$
	\item Binomio: Es un polinomio que consta de dos términos, por ejemplo: $-2x^2y^3 - 3x^2y^2$
	\item Trinomio: Es un polinomio que consta de tres términos, por ejemplo: $-2x^2y^3 - 3x^2y^2 + 4x^2y^1$
	\item Polinomio: Es un polinomio que consta de mas de tres términos, por ejemplo: $-2x^2y^3 - 3x^2y^2 + 4x^2y^1 + 5x^2y^0$
\end{itemize}

\subsubsection{Raíces y potencias con exponente racional}
Ejemplos:

\begin{center}
$\sqrt{x} = x^{\frac{1}{2}}$\\

$\sqrt[3]{x^5} = x^{\frac{5}{3}}$\\

$\sqrt[10]{x^{32}} = x^\frac{32}{10} = x^\frac{30}{10}x^\frac{2}{10} = x^3\sqrt[10]{x^2}$\\

$x^3 \times x^2 = x^5$\\

$x^2\sqrt{x^3} = x^2x^\frac{3}{2} = x^\frac{4}{2}x^\frac{3}{2} = x^{\frac{3}{2} + \frac{4}{2}} = x^\frac{7}{2} = \sqrt{x^7}$	
\end{center}
