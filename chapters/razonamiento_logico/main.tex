\section{Razonamiento lógico}

\subsection{Naturaleza y características del razonamiento}
\subsubsection{Elementos: materia, contenido y forma}

El razonamiento lógico es un proceso mental que nos permite llegar a conclusiones a partir de premisas. Para entender en profundidad el razonamiento, es importante conocer su naturaleza, características y los elementos que lo componen: materia, contenido y forma. Vamos a desglosar cada uno de estos aspectos.

\textbf{Materia}

La materia del razonamiento se refiere a los componentes básicos que se utilizan en el proceso de razonamiento, es decir, las proposiciones o premisas. Estas son las declaraciones que se consideran verdaderas o falsas y que sirven como punto de partida para el razonamiento.

\begin{itemize}
   \item \textbf{Proposición}: Una afirmación que puede ser verdadera o falsa. Por ejemplo, "Todos los seres humanos son mortales."
   \item \textbf{Premisas}: Proposiciones que sirven como base para llegar a una conclusión. Por ejemplo, en el silogismo "Todos los hombres son mortales. Sócrates es un hombre. Por lo tanto, Sócrates es mortal," las primeras dos proposiciones son las premisas.
\end{itemize}

\textbf{Contenido}
El contenido del razonamiento se refiere a la información específica y los conceptos incluidos en las proposiciones. Es el "qué" del razonamiento, el tema o asunto sobre el que se está razonando.

\begin{itemize}
   \item Conceptos: Las ideas o nociones que forman parte de las proposiciones. Por ejemplo, en la proposición "Todos los perros son mamíferos," los conceptos son "perros" y "mamíferos."
   \item Relaciones: Las conexiones entre los conceptos. En la misma proposición, la relación es de inclusión ("todos los... son...").
\end{itemize}

\textbf{Forma}

La forma del razonamiento se refiere a la estructura lógica y la disposición de las proposiciones para llegar a una conclusión. Es el "cómo" del razonamiento, la manera en que las proposiciones están organizadas.

\begin{itemize}
   \item Estructura: La disposición de premisas y conclusión. Por ejemplo, en el silogismo "Todos los A son B. Todos los B son C. Por lo tanto, todos los A son C," la estructura lógica es clara y válida.
   \item Reglas de Inferencia: Los principios que guían el paso de las premisas a la conclusión. Por ejemplo, el modus ponens es una regla de inferencia que dice que si A→B y A son verdaderos, entonces B debe ser verdadero.
         
\end{itemize}

\textbf{Ejemplos}
\begin{itemize}
   \item Materia: Las proposiciones "Todos los gatos son mamíferos" y "Todos los mamíferos son animales."
   \item Contenido: Los conceptos "gatos," "mamíferos," y "animales," y las relaciones de inclusión entre ellos.
   \item Forma: La estructura lógica del silogismo: Premisa 1($A \rightarrow B$), Premisa 2($B \rightarrow C$), y la conclusión ($A \rightarrow C$).
\end{itemize}

\subsubsection{Premisas y conclusión}

Las premisas son afirmaciones que se utilizan como punto de partida en un razonamiento lógico para llegar a una conclusión. Son las bases sobre las cuales se construye un argumento. Una premisa puede ser verdadera o falsa y su relación con otras premisas y la conclusión es lo que determina la validez del argumento.

\textbf{Tipos de premisas}

\begin{itemize}
   \item Premisas Empíricas:
         Basadas en la observación y la experiencia. Se refieren a hechos observables en el mundo real.
         
         Ejemplo: ''El cielo es azul durante el día."
   \item Premisas Analíticas:
         Basadas en la lógica y la definición de términos. Son verdaderas por el significado de las palabras involucradas.
         
         Ejemplo: "Todos los solteros no están casados."
   \item Premisas Normativas:
         Basadas en normas, valores o criterios subjetivos. Su verdad depende de un marco normativo o ético.
         
         Ejemplo: ''Es incorrecto mentir."
   \item Premisas Causales:
         Establecen una relación de causa y efecto entre dos eventos.
         
         Ejemplo: "Si llueve, el suelo se moja."
   \item Premisas Condicionales:
         Plantean una hipótesis en forma de "si... entonces...".
         
         Ejemplo: "Si estudio, aprobaré el examen."
   \item Premisas Universales:
         Hacen afirmaciones generales que se aplican a todos los casos en una categoría.
         
         Ejemplo: "Todos los mamíferos tienen corazón."
   \item Premisas Particulares:
         Se refieren a casos específicos dentro de una categoría más amplia.
         
         Ejemplo: ''Algunos perros son negros."
\end{itemize}

\textbf{Uso de Premisas en Argumentos}

\begin{itemize}
   \item \textbf{Premisas:}
         
         Todos los seres humanos son mortales. (Premisa Universal)
         
         Sócrates es un ser humano. (Premisa Particular)
         
   \item \textbf{Conclusión:}
         Sócrates es mortal.
         
   \item \textbf{Análisis:}
         
         Premisa 1 (Universal): Establece que todos los seres humanos pertenecen a la categoría de mortales.
         
         Premisa 2 (Particular): Sitúa a Sócrates dentro de la categoría de seres humanos.
         
         Conclusión: Deriva lógicamente de las premisas, aplicando la condición universal a un caso particular.
         
\end{itemize}

\subsubsection{Validez e invalidez}

La validez e invalidez son conceptos fundamentales en la lógica y el razonamiento. Estos conceptos ayudan a determinar si un argumento está bien estructurado y si sus conclusiones se derivan correctamente de sus premisas.

\textbf{Validez}

Un argumento es válido si, y solo si, la conclusión se sigue lógicamente de las premisas. En otras palabras, si las premisas son verdaderas, la conclusión debe ser necesariamente verdadera. La validez se refiere a la estructura lógica del argumento, no a la verdad de las premisas o la conclusión. Incluso si las premisas son falsas, un argumento puede ser válido si la conclusión se deriva lógicamente de ellas.

\textbf{Ejemplo de Argumento Válido:}

\textbf{Premisas:}
\begin{enumerate}
   \item Todos los mamíferos son animales.
   \item Todos los perros son mamíferos.
\end{enumerate}

\textbf{Conclusión}
Todos los perros son animales.

\textbf{Análisis}

\begin{itemize}
   \item La estructura del argumento es tal que si las premisas son verdaderas, la conclusión también debe ser verdadera.
   \item La validez depende únicamente de la forma lógica del argumento, que en este caso es correcta.
\end{itemize}

\textbf{Invalidez}

Un argumento es inválido si la conclusión no se sigue lógicamente de las premisas. Es decir, es posible que las premisas sean verdaderas y la conclusión sea falsa. Un argumento inválido tiene una estructura lógica defectuosa.

\textbf{Ejemplo de Argumento Inválido:}
\begin{itemize}
   \item Premisas:
         \begin{enumerate}
            \item Todos los gatos son mamíferos.
            \item Algunos animales son perros.
         \end{enumerate}
   \item Conclusión: Algunos gatos son perros.
   \item Análisis:
         \begin{itemize}
            \item Aunque las premisas son verdaderas, la conclusión no se deriva lógicamente de ellas.
            \item La estructura del argumento es defectuosa, lo que hace que el argumento sea inválido.
         \end{itemize}
\end{itemize}


\textbf{Diferencia Entre Verdad y Validez}

Es importante distinguir entre verdad y validez:

\begin{itemize}
   \item Verdad: Se refiere a si las proposiciones (premisas o conclusión) son factualmente correctas.
   \item Validez: Se refiere a la correcta estructura lógica del argumento, independientemente de si las proposiciones son verdaderas o falsas.
\end{itemize}

\textbf{Ejemplo de Argumento Válido con Premisas Falsas:}

\begin{itemize}
   \item Premisas:
         \begin{enumerate}
            \item Todos los marcianos son verdes.
            \item Juan es un marciano.
         \end{enumerate}
   \item Conclusión: Juan es verde.
   \item Análisis: Aunque las premisas son falsas, la conclusión se sigue lógicamente de ellas, por lo que el argumento es válido.
\end{itemize}


\textbf{Ejemplo de Argumento Inválido con Premisas Verdaderas:}


\begin{itemize}
   \item Premisas:
         \begin{enumerate}
            \item Todos los gatos son mamíferos.
            \item Algunos mamíferos son perros.
         \end{enumerate}
   \item Conclusión: Algunos gatos son perros.
   \item Análisis: Las premisas son verdaderas, pero la conclusión no se sigue lógicamente de ellas, por lo que el argumento es inválido.
\end{itemize}


\subsubsection{Relación de las premisas con la conclusión (implicación)}

La relación entre las premisas y la conclusión en un argumento se describe mediante el concepto de implicación lógica. Esta relación es fundamental para determinar la validez de un argumento. Aquí exploraremos en detalle cómo funcionan estas relaciones y cómo se determina si las premisas implican la conclusión.

\textbf{Implicación Lógica}

La implicación lógica se refiere a una relación entre proposiciones en la que, si las premisas son verdaderas, entonces la conclusión debe ser verdadera. En otras palabras, la verdad de las premisas garantiza la verdad de la conclusión.

\textbf{Notación}
En lógica, la implicación se denota comúnmente como $(P \rightarrow Q $, donde $P$  es una premisa o un conjunto de premisas, y $Q$ es la conclusión. Esto se lee como "P implica Q".

\textbf{Tipos de Implicación}

\begin{enumerate}
   \item Implicación Material:
         - Es la forma más común de implicación en lógica proposicional. Dice que \( P \rightarrow Q \) es verdadera excepto cuando \( P \) es verdadera y \( Q \) es falsa.
         - **Ejemplo:**
         - Premisa: "Si llueve (P), entonces la calle está mojada (Q)."
         - Conclusión: Si está lloviendo y la calle no está mojada, la implicación es falsa. En todos los demás casos, es verdadera.
         
   \item Implicación Formal:
         - Es la relación lógica que sostiene en virtud de la estructura del argumento, independientemente del contenido específico de las proposiciones.
         - **Ejemplo:**
         - Silogismo: "Todos los hombres son mortales (P1). Sócrates es un hombre (P2)."
         - Conclusión: "Sócrates es mortal (Q)."
         - La validez de este argumento se basa en su forma lógica.
\end{enumerate}


\textbf{Determinación de la Implicación}

Para determinar si las premisas implican la conclusión, se utilizan varias herramientas y métodos en lógica, entre los cuales destacan:

\begin{enumerate}
   \item Tablas de Verdad: Se utilizan para analizar todas las combinaciones posibles de verdad y falsedad de las premisas y la conclusión.
         
         Ejemplo:
         
         \begin{table}[ht]
            \centering
            \caption{Tabla de verdad de \( P \rightarrow Q \)}
            \begin{tabular}{|c|c|c|}
               \hline
               \textbf{P} & \textbf{Q} & \textbf{P $\rightarrow$ Q} \\
               \hline
               T          & T          & T                          \\
               \hline
               T          & F          & F                          \\
               \hline
               F          & T          & T                          \\
               \hline
               F          & F          & T                          \\
               \hline
            \end{tabular}
            \label{tab:truth-table}
         \end{table}
         
         \bigskip
         
         \textbf{Ejemplo:} Si la implicación \( P \rightarrow Q \) es verdadera en todas las combinaciones en las que \( P \) es verdadera, entonces la implicación es válida.
         
   \item Reglas de Inferencia: Conjunto de reglas que permiten derivar conclusiones válidas a partir de premisas.
         
         Ejemplos:
         
         \begin{itemize}
            \item Modus Ponens:\( P \rightarrow Q \), \( P \)  \( Q \).
            \item Modus Tollens: \( P \rightarrow Q \), \( \neg Q \)  \( \neg P \).
         \end{itemize}
         
   \item Silogismos: Forma de argumento deductivo que aplica una estructura lógica predeterminada.
         
         Ejemplo:
         \begin{itemize}
            \item Premisa Mayor: Todos los mamíferos son animales.
            \item Premisa Menor: Todos los perros son mamíferos.
            \item Conclusión: Todos los perros son animales.
         \end{itemize}
\end{enumerate}

\textbf{Ejemplos Prácticos de Implicación}

Ejemplo 1: Argumento Válido

\begin{itemize}
   \item Premisas:
         \begin{enumerate}
            \item Si llueve, entonces la calle está mojada.
            \item Está lloviendo.
         \end{enumerate}
   \item Conclusión: La calle está mojada.
   \item Análisis: Este argumento utiliza el modus ponens. Si las premisas son verdaderas, la conclusión debe ser verdadera.
\end{itemize}

Ejemplo 2: Argumento Inválido

\begin{itemize}
   \item Premisas:
         \begin{enumerate}
            \item Si estudio, aprobaré el examen.
            \item No aprobé el examen.
         \end{enumerate}
   \item Conclusión: No estudié.
   \item Análisis: Este argumento pretende utilizar una forma incorrecta del modus tollens. No aprobar el examen no implica necesariamente que no estudiaste, ya que puede haber otras razones.
\end{itemize}


La implicación lógica es una herramienta esencial en el razonamiento deductivo. Establece una relación entre premisas y conclusión que garantiza que si las premisas son verdaderas, la conclusión también debe ser verdadera. Comprender y utilizar correctamente la implicación lógica es clave para construir argumentos válidos y sólidos.

\subsection{Inferencias mediatas e inmediatas}

Las inferencias son procesos mentales mediante los cuales derivamos una conclusión a partir de una o más premisas. En lógica, se pueden clasificar en inferencias mediatas e inferencias inmediatas, dependiendo de cómo se llega a la conclusión. Vamos a explorar ambos tipos en detalle.

\subsubsection{Inferencias Inmediatas}

Una inferencia inmediata es un tipo de razonamiento en el cual se deriva una conclusión directamente de una sola premisa, sin necesidad de pasos intermedios adicionales.

\textbf{Tipos de Inferencias Inmediatas}

\begin{enumerate}
   \item Conversión: Consiste en intercambiar el sujeto y el predicado de una proposición.Por ejemplo:
         
         Premisa: "Todos los perros son mamíferos."
         
         Conversión: "Algunos mamíferos son perros."
         
         Nota: La conversión total es válida solo para proposiciones universales afirmativas (\( A \rightarrow I \)) y particulares afirmativas (\( I \rightarrow I \)).
         
   \item Obversión: Consiste en cambiar la cualidad de la proposición (de afirmativa a negativa o viceversa) y reemplazar el predicado por su complemento. Por ejemplo:
         
         Premisa: "Todos los perros son mamíferos."
         
         Obversión: "Ningún perro es no-mamífero."
         
         Nota: La obversión es válida para todos los tipos de proposiciones (\( A \rightarrow E \), \( E \rightarrow A \), \( I \rightarrow O \), \( O \rightarrow I \)).
         
   \item Contraposición: Consiste en invertir y negar tanto el sujeto como el predicado de una proposición. Por ejemplo
         
         Premisa: "Todos los perros son mamíferos."
         
         Contraposición: "Todos los no-mamíferos son no-perros."
         
         Nota: La contraposición es válida principalmente para proposiciones universales afirmativas (\( A \rightarrow A \)) y universales negativas (\( E \rightarrow E \)).
         
\end{enumerate}

\subsubsection{Inferencias Mediatas}

Una inferencia mediata es un tipo de razonamiento en el cual se deriva una conclusión a partir de dos o más premisas, utilizando pasos intermedios. Este tipo de inferencia es común en los silogismos.

\textbf{Tipos de Inferencias Mediatas}

\begin{enumerate}
   \item Silogismo Categórico: Es un argumento que consta de tres proposiciones: dos premisas y una conclusión. Las premisas conectan tres términos: el término mayor, el término menor y el término medio. Por ejemplo:
         
         Premisa Mayor: "Todos los mamíferos son animales."
         
         Premisa Menor: "Todos los perros son mamíferos."
         
         Conclusión: "Todos los perros son animales."
         
   \item Silogismo Hipotético: Es un argumento en el que al menos una de las premisas es una proposición condicional (si... entonces...).Por ejemplo:
         
         Premisa Mayor: "Si llueve, entonces la calle está mojada."
         
         Premisa Menor: "Está lloviendo."
         
         Conclusión: "La calle está mojada."
         
         Nota: Utiliza el modus ponens (afirmación del antecedente).
         
   \item Silogismo Disyuntivo: Es un argumento que contiene una premisa disyuntiva (o... o...). Por ejemplo:
         
         Premisa Mayor: "O llueve o hace sol."
         
         Premisa Menor: "No está lloviendo."
         
         Conclusión: "Hace sol."
         
         Nota: Utiliza el modus tollens (negación del consecuente).
\end{enumerate}

\subsubsection{Conversión simple}

La conversión simple es un tipo de inferencia inmediata en la que se intercambian el sujeto y el predicado de una proposición sin cambiar su valor de verdad. Este tipo de conversión es válida para proposiciones universales afirmativas (A) y particulares afirmativas (I). Ejemplos:

\textbf{Universal Afirmativa (A):}

Premisa: "Todos los perros son mamíferos."

Conversión Simple: ''Algunos mamíferos son perros."

\textbf{Particular Afirmativa (I):}

Premisa: ''Algunos mamíferos son perros."

Conversión Simple: ''Algunos perros son mamíferos."

\subsubsection{Conversión por accidente}

La conversión por accidente, también conocida como conversión per accidens, implica la conversión de una proposición universal afirmativa (A) en una proposición particular afirmativa (I) intercambiando el sujeto y el predicado. Ejemplo:

\textbf{Universal Afirmativa (A):}
Premisa: "Todos los gatos son animales."
Conversión por Accidente: "Algunos animales son gatos."

\subsubsection{Subalternación}
2.2.3. Subalternación
La subalternación es una inferencia inmediata que se refiere a la relación entre proposiciones universales y sus correspondientes proposiciones particulares. Si una proposición universal es verdadera, entonces la proposición particular correspondiente también es verdadera. Esto aplica tanto a afirmativas (A e I) como a negativas (E y O). Ejemplos:

\textbf{De Universal Afirmativa (A) a Particular Afirmativa (I):}

Premisa: "Todos los perros son mamíferos."

Subalternación: "Algunos perros son mamíferos."

\textbf{De Universal Negativa (E) a Particular Negativa (O):}

Premisa: "Ningún gato es un reptil."

Subalternación: "Algunos gatos no son reptiles."

\subsubsection{Contraposición}
La contraposición es una inferencia inmediata en la que se invierte y se niega tanto el sujeto como el predicado de una proposición. Este tipo de inferencia es válida principalmente para proposiciones universales afirmativas (A) y universales negativas (E). Ejemplos:

\textbf{Universal Afirmativa (A):}

Premisa: "Todos los perros son mamíferos."

Contraposición: "Todos los no-mamíferos son no-perros."

\textbf{Universal Negativa (E):}

Premisa: "Ningún gato es un reptil."

Contraposición: "Ningún no-reptil es un no-gato."

\subsubsection{Resumen}

\textbf{Conversión Simple:} Intercambia sujeto y predicado de una proposición. Válido para (A) y (I).

\textbf{Conversión por Accidente:} Convierte una proposición universal afirmativa (A) en particular afirmativa (I).

\textbf{Subalternación:} Relación entre proposiciones universales y sus correspondientes particulares. Si (A) es verdadera, entonces (I) también lo es; si (E) es verdadera, entonces (O) también lo es.

\textbf{Contraposición:} Invierte y niega tanto el sujeto como el predicado de una proposición. Válido para (A) y (E).

\subsection{Clases de razonamientos o inferencias mediatas}

Las inferencias mediatas son procesos de razonamiento que implican la derivación de conclusiones a partir de múltiples premisas. Aquí exploramos varias clases de inferencias mediatas: la deducción, la inducción, la analogía, la estadística o probabilidad, los métodos de Mill, y la inducción en la investigación científica.

\subsubsection{La deducción}

La deducción es un tipo de razonamiento en el que la conclusión se deriva necesariamente de las premisas. Si las premisas son verdaderas y el razonamiento es válido, la conclusión debe ser verdadera. Por ejemplo:

Premisa 1: Todos los humanos son mortales.

Premisa 2: Sócrates es humano.

Conclusión: Sócrates es mortal.

\textbf{Características:}

La conclusión sigue necesariamente de las premisas.

Si las premisas son verdaderas, la conclusión debe ser verdadera.

Es un proceso de razonamiento de lo general a lo particular.

\subsubsection{La inducción}
La inducción es un tipo de razonamiento en el que se deriva una conclusión general a partir de casos específicos observados. La conclusión no es garantizada, pero se considera probable basándose en las evidencias. Por ejemplo:

Observación 1: El sol ha salido por el este todos los días hasta ahora.

Conclusión: El sol saldrá por el este mañana.

\textbf{Características:}

La conclusión es probable, no necesariamente verdadera.

Se basa en la observación de casos particulares.

Es un proceso de razonamiento de lo particular a lo general.

\subsubsection{La analogía}
El razonamiento por analogía implica derivar una conclusión basándose en la similitud entre dos casos. Se asume que si dos cosas son similares en algunos aspectos, serán similares en otros. Por ejemplo:

Caso 1: Las células animales tienen mitocondrias para producir energía.

Caso 2: Las células vegetales también tienen mitocondrias para producir energía.

\textbf{Características:}

La conclusión se basa en la similitud entre dos casos.

No garantiza la verdad de la conclusión, solo sugiere una alta probabilidad.

Útil en situaciones donde no hay información directa.

\subsubsection{La estadística o probabilidad}
El razonamiento estadístico o probabilístico implica derivar conclusiones basándose en datos cuantitativos y probabilidades. Se utiliza para hacer inferencias sobre una población basándose en una muestra. Por ejemplo:

Premisa: El 90\(\%\) de los estudiantes en una encuesta prefieren clases online.

Conclusión: Probablemente, la mayoría de los estudiantes prefieren clases online.

\textbf{Características:}

La conclusión es probable, no certera.

Basada en el análisis de datos y probabilidades.

Utiliza técnicas estadísticas para inferir conclusiones.

\subsubsection{Los métodos de Mill}
John Stuart Mill propuso cinco métodos para establecer relaciones causales. Estos métodos son formas de razonamiento inductivo.

\begin{enumerate}
   \item Método de Concordancia: Si dos o más casos del fenómeno que se investiga tienen una sola circunstancia en común, esa circunstancia es la causa (o el efecto). Por ejemplo:
         
         Todos los pacientes que tomaron un medicamento específico se recuperaron. La recuperación es probablemente causada por el medicamento.
         
   \item Método de Diferencia: Si un caso en el que ocurre el fenómeno y un caso en el que no ocurre tienen todas las circunstancias en común excepto una, esa circunstancia es la causa (o el efecto). Por ejemplo:
         
         Un paciente toma un medicamento y se recupera. Otro paciente similar no toma el medicamento y no se recupera. La diferencia es el medicamento, que es la probable causa de la recuperación.
         
   \item Método Conjunto de Concordancia y Diferencia: Combina los dos primeros métodos para reforzar la inferencia causal.
   \item Método de los Residuos: Si se restan de un fenómeno las partes que se deben a ciertas causas, el residuo del fenómeno es causado por las circunstancias restantes. Por ejemplo:
         
         Se sabe que los efectos A y B se deben a causas conocidas. Si se observa un efecto C no explicado, se busca la causa de C en las circunstancias restantes.
         
   \item Método de las Variaciones Concomitantes: Si una variación en un fenómeno causa una variación en otro fenómeno, entonces están casualmente relacionados. Por ejemplo:
         
         A medida que se incrementa la dosis de un medicamento, la recuperación del paciente mejora. La variación en la dosis está causalmente relacionada con la variación en la recuperación.
\end{enumerate}

\subsubsection{La inducción en la investigación científica}
La inducción en la investigación científica es el proceso de desarrollar teorías generales a partir de observaciones y experimentos específicos. Este tipo de inducción es fundamental para el método científico. Por ejemplo:

\begin{itemize}
   \item Observación: Varios experimentos muestran que las plantas necesitan luz para realizar la fotosíntesis.
   \item Conclusión: Todas las plantas necesitan luz para realizar la fotosíntesis.
\end{itemize}

\textbf{Características:}

\begin{itemize}
   \item Basada en la observación sistemática y la experimentación.
   \item Utiliza la recopilación de datos empíricos para desarrollar teorías generales.
   \item Fundamental para la formulación de hipótesis y teorías científicas. 
\end{itemize}

\subsubsection{Resumen}

\begin{itemize}
   \item Deducción: Deriva conclusiones necesarias a partir de premisas generales.
   \item Inducción: Deriva conclusiones generales a partir de observaciones específicas.
   \item Analogía: Deriva conclusiones basadas en la similitud entre casos.
   \item Estadística o Probabilidad: Deriva conclusiones basadas en datos cuantitativos y probabilidades.
   \item Métodos de Mill: Proporcionan formas estructuradas de establecer relaciones causales.
   \item Inducción en la Investigación Científica: Utiliza observación y experimentación para desarrollar teorías generales.
\end{itemize}
