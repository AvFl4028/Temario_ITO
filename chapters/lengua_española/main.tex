\section{Lengua española}
\subsection{Funciones de la lengua}

Las funciones de la lengua se refieren a los diferentes usos que tiene el lenguaje en la comunicación. Cada función destaca un aspecto diferente del acto comunicativo. El lingüista Roman Jakobson identificó seis funciones principales, que se detallan a continuación:

\begin{enumerate}
      \item \textbf{Función referencial (o representativa)}
            \begin{itemize}
                  \item Propósito: Transmitir información subjetiva sobre la realidad.
                  \item Enfoque: El sujeto del mensaje.
                  \item Ejemplo: ''El agua hierve a 100 grados Celsius.''
                  \item Características: Predomina en textos científicos, noticias, informes, y cualquier comunicación que busca informar sobre hechos verificables.
            \end{itemize}

      \item \textbf{Función emotiva (o expresiva)}
            \begin{itemize}
                  \item Propósito: Expresar emociones, sentimientos o estados de ánimo del emisor.
                  \item Enfoque: El emisor del mensaje.
                  \item Ejemplo: ''¡Estoy tan feliz hoy!''
                  \item Características: Uso frecuente de exclamaciones, interjecciones y adjetivos que reflejan sentimientos. Común en literatura, diarios personales y conversaciones informales.
            \end{itemize}

      \item \textbf{Función conativa (o apelativa)}
            \begin{itemize}
                  \item Propósito: Influir en el comportamiento del receptor o dirigir su atención.
                  \item Ejemplo: ''Cierra la ventana, por favor.''
                  \item Características: Uso de imperativos, preguntas directas y llamados a la acción. Común en publicidad, discursos políticos y directrices.
            \end{itemize}

      \item \textbf{Función fática}
            \begin{itemize}
                  \item Propósito: Establecer, mantener o interrumpir la comunicación.
                  \item Enfoque: El canal de comunicación.
                  \item Ejemplo: ''Hola, ¿me escuchas?''
                  \item Características: Incluye saludos, despedidas, comprobaciones de funcionamiento del canal. Es esencial en conversaciones telefónicas y chats.
            \end{itemize}

      \item \textbf{Función metalingüística}
            \begin{itemize}
                  \item Propósito: Explicar o aclarar aspectos del propio lenguaje.
                  \item Enfoque: El código del mensaje.
                  \item Ejemplo: ''La palabra 'casa' se escribe con 'c'.''
                  \item Características: Frecuente en diccionarios, gramáticas y en contextos educativos donde se explica el uso del idioma.
            \end{itemize}

      \item \textbf{Función poética}
            \begin{itemize}
                  \item Propósito: Enfatizar la forma del mensaje, buscando belleza, ritmo o creatividad.
                  \item Enfoque: El mensaje en sí mismo.
                  \item Ejemplo: ''En el silencio sólo se escuchaba un susurro de abejas que sonaba.''
                  \item Características: Uso de recursos estilísticos como metáforas, rimas y figuras literarias. Predomina en poesía, literatura y publicidad.
            \end{itemize}
\end{enumerate}


\subsubsection{Ejemplos Aplicados}
\begin{itemize}
      \item \textbf{Función referencial:} ''Mañana habrá una conferencia sobre cambio climático a las 10 AM.''
      \item \textbf{Función emotiva:} ''¡Qué emoción ver esta obra de arte tan impresionante!''
      \item \textbf{Función conativa:} ''Necesitas completar el formulario antes del viernes.''
      \item \textbf{Función fática:} ''¿Hola? ¿Estás ahí?''
      \item \textbf{Función metalingüística:} ''El término 'sinónimo' se refiere a una palabra que tiene un significado similar a otra.''
      \item \textbf{Función poética:} ''El cielo, manto azul, abrigaba sus sueños.''
\end{itemize}

Cada una de estas funciones puede predominar en diferentes contextos comunicativos, aunque en una conversación real, a menudo se entrelazan y coexisten.

Si tienes alguna duda o necesitas más detalles sobre algún aspecto específico de las funciones de la lengua, no dudes en decírmelo.

\subsection{Connotación y denotación}

\textbf{Denotación} y \textbf{connotación} son dos aspectos del significado de las palabras. Comprender la diferencia entre estos términos es fundamental para una interpretación precisa del lenguaje y de los textos.

\subsubsection{Denotación}
La \textbf{denotación} se refiere al significado literal, objetivo y universal de una palabra. Es el significado que se encuentra en los diccionarios y que es común a todos los hablantes de una lengua. La denotación no cambia según el contexto ni las emociones de los hablantes. Por ejemplo: La palabra ''perro''.

\textbf{Denotación}: Animal mamífero de la familia Canidae, domesticado y que suele ser un compañero de los humanos.

La denotación es el significado básico y esencial de una palabra, sin matices adicionales ni subjetividades.

\subsubsection{Connotación}

La \textbf{connotación}, por otro lado, se refiere a los significados adicionales, subjetivos y emocionales que una palabra puede tener. Estas asociaciones pueden variar según el contexto, la cultura y la experiencia personal del hablante o del oyente. La connotación añade capas de significado que pueden enriquecer o alterar la interpretación de la palabra. Por ejemplo: La palabra ''perro''.

\textbf{Connotaciones}:

En algunos contextos, puede connotar lealtad y amistad.

En otros, puede tener connotaciones negativas si se utiliza como insulto o para describir a alguien como agresivo o sucio.

Las connotaciones pueden ser positivas, negativas o neutrales, y dependen en gran medida del contexto cultural y personal.

Ejemplos Comparativos

\begin{itemize}
      \item \textbf{Palabra: Casa}
      \item \textbf{Denotación}: Edificio para habitar.
      \item \textbf{Connotaciones}: Puede evocar sentimientos de seguridad, calor, familia o, en otros casos, puede connotar monotonía o encierro.
\end{itemize}

\begin{itemize}
      \item \textbf{Palabra: Serpiente}
      \item \textbf{Denotación}: Reptil de cuerpo alargado y sin extremidades.
      \item  \textbf{Connotaciones}: En muchas culturas, puede simbolizar traición o peligro, pero en otras, puede representar sabiduría o transformación.
\end{itemize}

\begin{itemize}
      \item \textbf{Palabra: Rojo}
      \item \textbf{Denotación}: Color primario con una longitud de onda de aproximadamente 620-750 nm.
      \item \textbf{Connotaciones}: Puede asociarse con pasión, amor, peligro o ira, dependiendo del contexto.
\end{itemize}

\subsubsection{Aplicaciones Prácticas}

\textbf{Publicidad y marketing}: Las connotaciones se usan para evocar emociones específicas en el público. Por ejemplo, usar palabras como ''hogar'' en lugar de ''casa'' para crear una sensación de calidez y pertenencia.

\textbf{Análisis literario}: Identificar las connotaciones en un poema o una novela puede revelar los sentimientos y las intenciones del autor.

\textbf{Comunicación interpersonal}: Ser consciente de las connotaciones ayuda a evitar malentendidos y a comunicarse de manera más efectiva y sensible.

\subsection{Homónimos, sinónimos y antónimos}

\subsubsection{Homónimos}

Los \textbf{homónimos} son palabras que se pronuncian o se escriben de la misma manera pero tienen significados diferentes. Los homónimos pueden subdividirse en:

\begin{itemize}
      \item \textbf{Homógrafos}: Palabras que se escriben y se pronuncian igual, pero tienen diferentes significados.

            \textbf{Ejemplo}:
            \begin{enumerate}
                  \item ''Banco'' (entidad financiera)
                  \item ''Banco'' (asiento para sentarse)
            \end{enumerate}

      \item \textbf{Homófonos}: Palabras que se pronuncian igual pero se escriben de manera diferente y tienen significados distintos.

            \textbf{Ejemplo}:
            \begin{enumerate}
                  \item ''Hola'' (saludo)
                  \item ''Ola'' (onda en el agua)
            \end{enumerate}
\end{itemize}

\subsubsection{Sinónimos}

Los \textbf{sinónimos} son palabras que tienen significados iguales o muy similares. Se utilizan para evitar la repetición de palabras y para enriquecer el lenguaje.

\textbf{Ejemplo:}

\begin{itemize}
      \item ''Casa'' y ''hogar''
      \item ''Rápido'' y ''veloz''
      \item ''Feliz'' y ''contento''
\end{itemize}

\subsubsection{Antónimos}

Los \textbf{antónimos} son palabras que tienen significados opuestos o contrarios. Ayudan a expresar contrastes y matices en el lenguaje.

\textbf{Ejemplo:}

\begin{itemize}
      \item ''Alto'' y ''bajo''
      \item ''Día'' y ''noche''
      \item ''Frío'' y ''caliente''
\end{itemize}

\subsubsection{Ejemplos Aplicados}

\textbf{Homónimos}
\begin{itemize}
      \item \textbf{Homógrafos:}
            \begin{enumerate}
                  \item ''Vino'' (bebida alcohólica) y ''vino'' (verbo venir en pasado)
                  \item ''Llama'' (animal) y ''llama'' (fuego)
            \end{enumerate}
      \item \textbf{Homófonos:}
            \begin{enumerate}
                  \item ''Bello'' (hermoso) y ''vello'' (pelo)
                  \item ''Cazar'' (atrapar animales) y ''casar'' (unir en matrimonio)
            \end{enumerate}
\end{itemize}

\textbf{Sinónimos}

\begin{itemize}
      \item ''Niño'' y ''chico''
      \item ''Enojado'' y ''molesto''
      \item ''Difícil'' y ''complicado''
\end{itemize}

\textbf{Antónimos}

\begin{itemize}
      \item ''Blanco'' y ''negro''
      \item ''Rápido'' y ''lento''
      \item ''Grande'' y ''pequeño''
\end{itemize}

\textbf{Importancia en el Lenguaje}

\begin{itemize}
      \item \textbf{Homónimos:} Desafían al hablante y al oyente a usar y comprender el contexto para interpretar el significado correcto.
            Pueden generar juegos de palabras y ambigüedades deliberadas en la literatura y el humor.
      \item \textbf{Sinónimos:} Enriquecen el lenguaje permitiendo una mayor variedad de expresiones.
            Evitan la monotonía en el habla y la escritura, ofreciendo alternativas que matizan el mensaje.
      \item \textbf{Antónimos:}

            Facilitan la expresión de contrastes y diferencias.
            Ayudan a definir y aclarar conceptos mediante la oposición.
\end{itemize}

\textbf{Ejercicio Práctico}

Para consolidar estos conceptos, podrías realizar un ejercicio sencillo:

\begin{itemize}
      \item \textbf{Encuentra homónimos}: Identifica dos palabras homónimas en una oración.
      \item \textbf{Busca sinónimos}: Encuentra sinónimos para una lista de palabras comunes.
      \item \textbf{Empareja antónimos}: Asocia palabras con sus antónimos correspondientes.
\end{itemize}

\subsection{El enunciado}

Un \textbf{enunciado} es una unidad mínima de comunicación que tiene sentido completo y puede estar constituido por una o varias palabras. Los enunciados son fundamentales en la comunicación porque permiten transmitir información, expresar emociones, realizar preguntas, dar órdenes, entre otros.

\subsubsection{Enunciado bimembre u oración}

Un \textbf{enunciado bimembre} es aquel que se puede dividir en dos partes principales: el sujeto y el predicado. Este tipo de enunciado también se conoce como **oración**.

\textbf{Estructura de una Oración}

\begin{enumerate}
      \item \textbf{Sujeto}: Es la parte de la oración que indica quién realiza la acción o de quién se dice algo. Generalmente, el sujeto es un sustantivo, un pronombre o una frase sustantiva. Por ejemplo: ''El gato'' en la oración ''El gato duerme.''
      \item \textbf{Predicado}: Es la parte de la oración que indica la acción realizada por el sujeto o lo que se dice del sujeto. El predicado siempre contiene un verbo. Por ejemplo: ''duerme'' en la oración ''El gato duerme.''
\end{enumerate}

\subsubsection{Elementos de la oración: sujeto y predicado}

\textbf{Sujeto:} El sujeto puede ser explícito o implícito (tácito), y tiene un núcleo que suele ser un sustantivo o un pronombre.

\begin{itemize}
      \item \textbf{Sujetos explícitos}: Son aquellos que se especifican en la oración. Por ejemplo: ''María canta una canción.'' (''María'' es el sujeto)
      \item \textbf{Sujetos implícitos}: Son aquellos que se sobreentiende que son ''él/ella'' o ''usted''. Por ejemplo: ''Canta una canción.'' (El sujeto no está presente en la oración pero se sobreentiende que es ''él/ella'' o ''usted'')
\end{itemize}

El sujeto puede estar acompañado de modificadores que lo amplían o precisan:

\begin{itemize}
      \item \textbf{Modificadores directos}: Adjetivos que califican directamente al sustantivo. Por ejemplo: ''El gato negro''
      \item \textbf{Modificadores indirectos}: Grupos preposicionales que modifican al sustantivo. Por ejemplo: ''El gato de mi vecino''
\end{itemize}

\textbf{Predicado}: El predicado también tiene un núcleo, que es siempre un verbo, y puede contener otros elementos que complementan el significado del verbo:

\begin{itemize}
      \item \textbf{Predicado nominal:} Formado por un verbo copulativo (ser, estar, parecer) y un atributo. Por ejemplo: ''María es médica.''
      \item \textbf{Predicado verbal:} Formado por un verbo predicativo y sus complementos. Por ejemplo: ''María canta una canción.''
\end{itemize}

Elementos que pueden acompañar al predicado:
\begin{itemize}
      \item \textbf{Complemento directo}: Recibe directamente la acción del verbo. Por ejemplo: ''Maria lee un libro.'' (''un libro'' es el complemento directo).
      \item \textbf{Complemento indirecto}: Indica el destinatario o beneficiario de la acción. Por ejemplo: ''María escribe una carta a su amigo.'' (''a su amigo'' es el complemento indirecto).
      \item \textbf{Complemento circunstancial}: Indica las circunstancias de la acción (lugar, tiempo, modo, etc.). Por ejemplo: ''María canta en el parque.'' (''en el parque'' es el complemento circunstancial de lugar).
\end{itemize}

Ejemplos

\begin{itemize}
      \item \textbf{Oración simple}: ''El perro ladra.''
            \begin{itemize}
                  \item Sujeto: ''El perro''
                  \item Predicado: ''ladra''
            \end{itemize}
      \item \textbf{Oración compuesta}: ''El perro ladra y el gato duerme.''
            \begin{itemize}
                  \item Primera oración: ''El perro ladra.''
                  \item Segunda oración: ''El gato duerme.''
                  \item Sujeto: ''El perro''
                  \item Predicado: ''ladra''
                  \item Sujeto: ''El gato''
                  \item Predicado: ''duerme''
            \end{itemize}
\end{itemize}

\subsection{Signos de puntuación}

\begin{enumerate}
      \item Punto (.)
            \begin{itemize}
                  \item Función: Marca el final de una oración declarativa o imperativa.
                  \item Ejemplo: ''El tren llegó a la estación.''
            \end{itemize}
      \item Coma (,)
            \begin{itemize}
                  \item Función: Indica pausas cortas en una oración, separa elementos de una lista, y separa cláusulas o frases dentro de una oración.
                  \item Ejemplo: ''María fue al mercado, compró frutas y verduras, y regresó a casa.''
            \end{itemize}
      \item Punto y Coma (;)
            \begin{itemize}
                  \item Función: Marca una pausa mayor que la coma pero menor que el punto. Se utiliza para separar oraciones relacionadas entre sí y que podrían funcionar como oraciones independientes.
                  \item Ejemplo: ''Estudia mucho; aprobarás el examen.''
            \end{itemize}
      \item Dos Puntos (:)
            \begin{itemize}
                  \item Función: Introducen una enumeración, una cita o un diálogo, y preceden a una explicación o a una lista.
                  \item  Ejemplo: ''Tenía tres opciones: quedarse en casa, ir al cine o salir a cenar.''
            \end{itemize}
      \item Puntos Suspensivos (…)
            \begin{itemize}
                  \item Función: Introducen una enumeración, una cita o un diálogo, y preceden a una explicación o a una lista.
                  \item Ejemplo: ''Tenía tres opciones: quedarse en casa, ir al cine o salir a cenar.''
            \end{itemize}
      \item Signos de interrogación (¿?)
            \begin{itemize}
                  \item Función: Indican que una pregunta ha sido formulada.
                  \item Ejemplo: ''¿Cómo estás?''
            \end{itemize}
      \item Signos de Exclamación (!)
            \begin{itemize}
                  \item Función: Indican emoción, sorpresa, admiración o enfatizan una idea.
                  \item Ejemplo: ''¡Qué hermoso día hace hoy!''
            \end{itemize}
      \item Comillas
            \begin{itemize}
                  \item Función: Encierran citas textuales, títulos de obras, palabras o expresiones destacadas, y se usan para indicar ironía o distanciamiento de una palabra.
                  \item Ejemplo: ''Dijo: 'Volveré mañana'.''
            \end{itemize}
      \item Paréntesis
            \begin{itemize}
                  \item Función: Encierran información adicional o explicativa que no es esencial para la comprensión del texto principal.
                  \item Ejemplo: ''La ciudad (que había sido fundada en el siglo XV) es famosa por su arquitectura.''
            \end{itemize}
\end{enumerate}

\textbf{Importancia de los Signos de Puntuación}

\begin{itemize}
      \item Claridad y Coherencia: Ayudan a estructurar y dar sentido al texto, evitando ambigüedades y facilitando la comprensión.
      \item Ritmo y Entonación: Indican pausas y entonación adecuada para la lectura fluida y natural.
      \item Énfasis y Expresión: Permiten resaltar ideas, expresar emociones y transmitir matices de significado.
      \item Correcta Interpretación: La ausencia o el uso incorrecto de los signos de puntuación puede alterar el sentido de una frase u oración.
\end{itemize}

\textbf{Ejercicio Práctico}

Un ejercicio útil para practicar el uso de los signos de puntuación es tomar un texto y modificar su significado al alterar la puntuación. También puedes escribir un texto propio y jugar con la ubicación de los signos para observar cómo cambia el sentido.

Si tienes alguna pregunta adicional o necesitas más ejemplos sobre el tema de los signos de puntuación, no dudes en preguntar.

\subsection{Las mayúsculas}

\textbf{Reglas de Uso de Mayúsculas}
\begin{enumerate}
      \item Al Inicio de Oración
            \begin{itemize}
                  \item Regla: La primera letra de cada oración se escribe con mayúscula.
                  \item Ejemplo: ''Hoy es un buen día para salir a pasear.''
            \end{itemize}
      \item Nombres Propios
            \begin{itemize}
                  \item Regla: Los nombres propios de personas, lugares, instituciones, empresas, marcas y títulos se escriben con mayúscula.
                  \item Ejemplo: ''Juan, Madrid, Universidad de Harvard, Google, El Quijote.''
            \end{itemize}
      \item Títulos
            \begin{itemize}
                  \item Regla: Los títulos de obras (libros, películas, obras de arte), así como los títulos honoríficos y los nombres de eventos importantes, se escriben con mayúscula.
                  \item Ejemplo: ''La Odisea, Doctor, Festival Internacional de Cine de Cannes.''
            \end{itemize}
      \item Abreviaturas
            \begin{itemize}
                  \item Regla: Las abreviaturas, acrónimos y siglas se escriben en mayúsculas.
                  \item Ejemplo: ''EE.UU. (Estados Unidos), ONU (Organización de las Naciones Unidas), CEO (Chief Executive Officer).''
            \end{itemize}
      \item Fechas y Meses
            \begin{itemize}
                  \item Regla: Los nombres de los meses y los días de la semana se escriben con mayúscula.
                  \item Ejemplo: ''Hoy es Lunes, 12 de Abril.''
            \end{itemize}
      \item Inicios de Línea
            \begin{itemize}
                  \item Regla: Después de un punto seguido, las palabras siguientes comienzan con mayúscula.
                  \item Ejemplo: ''Ella vive en París. Allí trabaja como arquitecta.''
            \end{itemize}
      \item Siguiendo Dos Puntos
            \begin{itemize}
                  \item Regla: Después de dos puntos, la primera letra de la oración siguiente se escribe con mayúscula si esta continúa la idea de la oración anterior o si se inicia una enumeración.
                  \item Ejemplo: ''Tengo tres opciones: Ir al cine, leer un libro o salir a pasear.''
            \end{itemize}
\end{enumerate}

\textbf{Importancia de las Mayúsculas}

\begin{itemize}
      \item Claridad: Ayudan a distinguir los títulos y nombres propios, lo que facilita la comprensión del texto.
      \item Respeto: El uso adecuado de mayúsculas para los nombres propios y títulos muestra respeto hacia las personas y organizaciones.
      \item Estilo y Formalidad: Contribuyen a la presentación formal y ordenada de un texto, especialmente en documentos oficiales y académicos.
\end{itemize}

\textbf{Ejercicio Práctico}

Puedes practicar el uso correcto de las mayúsculas escribiendo un breve texto y asegurándote de aplicar las reglas mencionadas. También puedes revisar textos existentes para identificar errores o para mejorar la presentación.

\subsection{Acentuación}

\textbf{Reglas de Acentuación}

\begin{enumerate}
      \item Acento en Palabras Agudas: Las palabras agudas llevan tilde cuando terminan en vocal, ''n'' o ''s'' y la última sílaba es acentuada.Por ejemplo: ''café'', ''jamón'', ''canción''.
      \item Acento en Palabras Graves o Llanas: Las palabras graves llevan tilde cuando terminan en consonante que no sea ''n'' o ''s'' y la penúltima sílaba es acentuada. Por ejemplo: ''lápiz'', ''cántaro'', ''médico''.
      \item Acento en Palabras Esdrújulas: Las palabras esdrújulas siempre llevan tilde en la vocal tónica. Por ejemplo: ''música'', ''fácilmente'', ''ántes''.
      \item Acento en Palabras Sobresdrújulas: Las palabras sobresdrújulas siempre llevan tilde en la vocal tónica. Por ejemplo: ''déjame'', ''explícamelo'', ''dímelo''.
      \item Diptongos e Hiatos: En los diptongos, formados por dos vocales juntas que se pronuncian en una misma sílaba, la tilde recae sobre la vocal cerrada (i, u) si la palabra lo requiere. Por ejemplo: ''baúl'', ''cuídate''. En los hiatos, formados por dos vocales que se pronuncian en sílabas distintas, la tilde recae sobre la vocal abierta (a, e, o) si la palabra lo requiere. Por ejemplo: ''país'', ''día'', ''oír''.
      \item Palabras monosílabas: Las palabras monosílabas llevan tilde si son tónicas y tienen acento diacrítico para diferenciar significados. Por ejemplo: ''sí'' (adverbio de afirmación) y ''si'' (conjunción condicional).
\end{enumerate}

\textbf{Importancia de la Acentuación}

\begin{itemize}
      \item Claridad y Comprensión: Los acentos ayudan a diferenciar significados entre palabras que se escriben de forma similar.
      \item Correcta Pronunciación: Indican qué sílaba se debe pronunciar con mayor énfasis, facilitando la correcta entonación del habla.
      \item Ortografía Correcta: La correcta acentuación es esencial para una ortografía precisa y adecuada.
\end{itemize}

\textbf{Ejercicio Práctico}

Puedes practicar la acentuación revisando textos escritos en español y prestando atención a las palabras que llevan tilde. También puedes practicar escribiendo palabras y aplicando las reglas de acentuación correspondientes.

\subsection{Ortografía}

\subsubsection{Reglas de Ortografía}

\begin{enumerate}
      \item Uso de las Letras
            \begin{itemize}
                  \item Se utiliza ''b'' después de ''m'' y ''v'', y en sílabas ''mb'' y ''bv''.
                  \item Ejemplo: ''umbrales'', ''subvención''.
                  \item Se utiliza ''v'' en lugar de ''b'' en palabras como ''vacaciones'', ''verdad''.
                  \item Ejemplo: ''vacaciones'', ''verdad''.
                  \item Se utiliza ''c'' antes de ''e'' e ''i'', y ''qu'' antes de ''e'' e ''i''.
                  \item Ejemplo: ''cielo'', ''queso''.
                  \item Se utiliza ''g'' antes de ''e'' e ''i'', y ''gu'' antes de ''e'' e ''i''.
                  \item Ejemplo: ''gente'', ''guiso''.
                  \item La ''h'' es muda en español, excepto en los casos de ''ch'' y ''nh''.
                  \item Ejemplo: ''chico'', ''baño''.
                  \item La ''ñ'' se utiliza en lugar de ''ni'' o ''n+y'' y en algunas palabras como ''mañana'', ''niño''.
            \end{itemize}
      \item Acentuación
            \begin{itemize}
                  \item Las reglas de acentuación indican en qué sílaba debe recaer el acento ortográfico en una palabra.
                  \item Ejemplo: ''rápido'', ''cántaro''.
                  \item Se utiliza el acento diacrítico en palabras monosílabas y en casos de palabras homógrafas que cambian de significado.
                  \item Ejemplo: ''tú'' (pronombre personal) y ''tu'' (posesivo).
            \end{itemize}
      \item Diptongos e Hiatos
            \begin{itemize}
                  \item Se forma un diptongo cuando hay una combinación de una vocal abierta (a, e, o) y una vocal cerrada (i, u) o dos vocales cerradas.
                  \item Ejemplo: ''aire'', ''cielo''.
                  \item Se forma un hiato cuando dos vocales que forman parte de la misma palabra se pronuncian en sílabas diferentes.
                  \item Ejemplo: ''río'', ''país''.
            \end{itemize}
      \item Homófonos y Homógrafos
            \begin{itemize}
                  \item Los homófonos son palabras que suenan igual pero se escriben de forma diferente y tienen significados distintos.
                  \item Ejemplo: ''bello'' (hermoso) y ''vello'' (pelo).
                  \item Los homógrafos son palabras que se escriben igual pero tienen significados diferentes.
                  \item Ejemplo: ''calle'' (vía pública) y ''calle'' (formar el verbo ''callar'').
            \end{itemize}
      \item Uso de Mayúsculas
            \begin{itemize}
                  \item Se utilizan mayúsculas al principio de una oración, en nombres propios, títulos, fechas, acrónimos, entre otros.
                  \item Ejemplo: ''Madrid'', ''Doctor'', ''España''.
            \end{itemize}
\end{enumerate}

\subsubsection{Importancia de la Ortografía}

\begin{itemize}
      \item Claridad y Precisión: Una correcta ortografía garantiza que el mensaje sea entendido de manera clara y precisa.
      \item Profesionalismo: Una buena ortografía es importante en el ámbito laboral y académico, ya que refleja el cuidado y la atención al detalle.
      \item Cultura y Educación: La ortografía es una parte esencial de la educación y la cultura, y contribuye a la correcta transmisión del conocimiento.
\end{itemize}

\subsubsection{Ejercicio Práctico}

Puedes practicar la ortografía revisando tus escritos, utilizando diccionarios y recursos en línea para consultar dudas ortográficas, y participando en ejercicios de práctica de ortografía.

\subsection{Comprensión de lectura}

La comprensión de lectura es la capacidad de entender y procesar la información contenida en un texto escrito. Es un proceso cognitivo complejo que implica decodificar el significado de las palabras, relacionarlas entre sí, identificar las ideas principales y secundarias, inferir significados implícitos, y hacer conexiones con conocimientos previos. Aquí te explico los aspectos fundamentales de la comprensión de lectura:

\subsubsection{Estrategias para Mejorar la Comprensión de Lectura}

\begin{enumerate}
      \item Antes de la Lectura
            \begin{itemize}
                  \item Activación de Conocimientos Previos: Reflexiona sobre lo que ya sabes acerca del tema del texto para activar tu esquema de conocimiento.
                  \item Establecimiento de Objetivos: Define qué esperas obtener de la lectura y qué información necesitas encontrar.
            \end{itemize}
      \item Durante la Lectura
            \begin{itemize}
                  \item Mantén la Concentración: Evita distracciones y mantén el enfoque en el texto.
                  \item Subrayado y Anotaciones: Subraya las ideas principales y haz anotaciones en los márgenes para resaltar puntos importantes o preguntas que surjan.
                  \item Monitoreo de la Comprensión: Asegúrate de entender lo que lees haciendo pausas para reflexionar sobre el significado de las palabras y las ideas.
            \end{itemize}
      \item Después de la Lectura
            \begin{itemize}
                  \item Resumen: Sintetiza las ideas principales del texto en tus propias palabras.
                  \item Evaluación: Reflexiona sobre lo que has aprendido y si has cumplido tus objetivos de lectura.
                  \item Hacer Conexiones: Relaciona la información del texto con tus experiencias personales, otros textos o el mundo que te rodea.
            \end{itemize}
\end{enumerate}

\subsubsection{Factores que Influyen en la Comprensión de Lectura}

\begin{enumerate}
      \item Vocabulario y Conocimiento del Tema: Cuanto más extenso sea tu vocabulario y conocimiento sobre el tema del texto, más fácil será comprenderlo.
      \item Habilidades de Decodificación: La capacidad para reconocer palabras y entender la estructura gramatical del texto es fundamental para la comprensión.
      \item Motivación y Actitud: La disposición mental y el interés en el tema pueden afectar significativamente tu comprensión y retención de la información.
      \item Estrategias de Comprensión: El uso de estrategias efectivas, como hacer predicciones, hacer preguntas y resumir, mejora la comprensión de lectura.
\end{enumerate}

\subsubsection{Importancia de la Comprensión de Lectura}
\begin{itemize}
      \item Desarrollo Académico: Es crucial para el éxito en la educación, ya que la mayoría del aprendizaje se realiza a través de la lectura.
      \item Competencia Profesional: Mejora tus habilidades de comunicación, análisis y toma de decisiones en el entorno laboral.
      \item Crecimiento Personal: Te permite expandir tus horizontes, adquirir nuevos conocimientos y desarrollar una mente crítica y reflexiva.
\end{itemize}

\subsubsection{Ejercicio Práctico}
Puedes practicar la comprensión de lectura eligiendo textos que te interesen y utilizando las estrategias mencionadas para mejorar tu comprensión. También puedes participar en grupos de lectura o discusión para compartir ideas y perspectivas sobre los textos.