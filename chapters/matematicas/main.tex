\section{Matemáticas}

\subsection{Productos Notables}
\subsubsection{Binomio de Newton}
\subsubsection{Simplificación de fracciones algebraicas}
\subsubsection{Operaciones con fracciones algebraicas}

\subsection{Ecuaciones}
\subsubsection{Ecuación, propiedades, clases}
\subsubsection{Ecuaciones de primer grado}
\subsubsection{Ecuaciones de segundo grado}

\subsection{Sistemas de ecuaciones}
\subsubsection{Sistemas de dos ecuaciones lineales con dos incógnitas}
\subsubsection{Sistemas de dos ecuaciones}

\subsection{Recta}
\subsubsection{Distancia entre dos puntos}
\subsubsection{Punto medio}
\subsubsection{Pendiente de una recta}
\subsubsection{Condiciones de paralelismo y perpendicularidad}

\subsection{Circunferencia}
\subsubsection{Circunferencia como lugar geométrico}
\subsubsection{Calcular la ecuación de una circunferencia con centro en el origen}
\subsubsection{Ecuación de la circunferencia forma general y forma canónica}
\subsubsection{Elementos de una circunferencia}

\subsection{Límites}
\subsubsection{Definición formal}
\subsubsection{Teoremas sobre límites}
\subsubsection{Evaluar límite}
\subsubsection{Límite indeterminado}

\subsection{La derivada}
\subsubsection{Definición de derivada}
\subsubsection{Derivar $x^3 - 3x^2 + x - 1$: Obtención de derivadas}
\subsubsection{Interpretación geométrica}
\subsubsection{Ecuación de la recta tangente y de la recta normal}
\subsubsection{Cálculo de la velocidad y aceleración de un móvil usando derivadas}


% https://www.youtube.com/watch?v=apPXOlZnRhg
% https://chatgpt.com/share/7be82482-f39d-47b7-9df9-f0c453cfdabb
% https://www.uaeh.edu.mx/docencia/P_Presentaciones/prepa3/2019/Coordenadas.pdf
% https://es.khanacademy.org/math/geometry/hs-geo-analytic-geometry/hs-geo-distance-and-midpoints/a/midpoint-formula
% https://es.khanacademy.org/math/algebra/x2f8bb11595b61c86:linear-equations-graphs/x2f8bb11595b61c86:slope/a/slope-review
% http://jfaustocmath.weebly.com/home/capitulo-iii-paralelismo-y-perpendicularidad#:~:text=Se%20dice%20que%20dos%20segmentos,si%20sus%20pendientes%20son%20iguales.&text=Tambi%C3%A9n%20se%20define%20como%20segmentos,de%20sus%20pendientes%20es%20%2D1.
% https://www.geogebra.org/m/fdqxnj4f
% https://www.uaeh.edu.mx/docencia/P_Presentaciones/prepa_ixtlahuaco/2017/geometria.pdf