\section{Matemáticas}

\subsection{Productos Notables}

Los productos notables son fórmulas algebraicas que permiten simplificar el proceso de multiplicación de ciertos polinomios. Aquí hay algunos ejemplos importantes:

\subsubsection{Binomio de Newton}

El Binomio de Newton se refiere a la expansión de una potencia de un binomio, expresado generalmente como $(a+b)^n$. La fórmula general del binomio de Newton es:

\begin{center}
    $\sum_{k=0}^n \binom{n}{k} a^{n - k} b^k$
\end{center}

Donde $\binom{n}{k}$ es el coeficiente binomial, calculado como:

\begin{center}
    $\binom{n}{k} = \frac{n!}{k!(n-k)!}$
\end{center}

Ejemplo:

Para expandir $(x + y)^3$:

\begin{center}
    $(x + y)^3 = \sum_{k=0}^3 \binom{3}{k}x^{3-k}y^k$\\
    $\binom{3}{0}x^3 + \binom{3}{1}x^2y + \binom{3}{2}xy^2 + \binom{3}{3}y^3$\\
    $= x^3 + 3x^2y + 3xy^2 + y^3$
\end{center}


\subsubsection{Simplificación de fracciones algebraicas}
Para simplificar fracciones algebraicas, se deben factorizar el numerador y el denominador y luego cancelar los factores comunes.

\textbf{Ejemplo:}

Simplificar $\frac{6x^2-12x}{3x}$

\begin{enumerate}
    \item Factorizar: $\frac{6x^2-12x}{3x} = \frac{6x(x-2)}{3x}$
    \item Simplificar la fracción: $\frac{6x(x-2)}{3x} = \frac{6x}{3x}(x-2) = 2(x-2)$
\end{enumerate}

\subsubsection{Operaciones con fracciones algebraicas}

Las operaciones con fracciones algebraicas incluyen suma, resta, multiplicación y división. A continuación, se presentan ejemplos para cada operación.

\textbf{Suma y resta}

Para sumar o restar fracciones algebraicas, es necesario encontrar un denominador común.

\textbf{Ejemplo:}

Sumar $\frac{2}{x} + \frac{3}{x^2}$:

\begin{enumerate}
    \item Encontrar el denominador común: $x^2$
    \item Reescribir las fracciones:
          $\\
              \frac{2}{x} = \frac{2x}{x^2}\\
              \frac{3}{x^2} = \frac{3}{x^2}
          $
    \item Sumar las fracciones: $\frac{2x}{x^2} + \frac{3}{x^2} = \frac{2x+3}{x^2}$
\end{enumerate}

\textbf{Multiplicación y División}

Multiplicar $\frac{2x}{3y} \cdot \frac{4y}{5x}$

$\frac{2x4y}{5x3y} = \frac{8xy}{15xy} = \frac{8}{15}$

Dividir $\frac{2x}{3y} \div \frac{4y}{5x}$

$\frac{2x}{3y} \div \frac{4y}{5x} = \frac{2x}{3y} \cdot \frac{5x}{4y} = \frac{10x^2}{12y^2} = \frac{5x^2}{6y^2}$

\subsection{Ecuaciones}

Una ecuación es una igualdad matemática que contiene una o más variables. Resolver una ecuación implica encontrar el valor o los valores de las variables que hacen que la igualdad sea verdadera.

\subsubsection{Ecuación, propiedades, clases}

\textbf{Propiedades de las ecuaciones}

Propiedad de la igualdad: Si $a=b$, entonces $a+c=b+c$ y $a-c=b-c$.

Propiedad de la multiplicación: Si $a=b$, entonces $ac=bc$.

Propiedad de la división: Si $a=b$ y $c \neq 0$, entonces $\frac{a}{c}=\frac{b}{c}$.

Propiedad distributiva: $a(b+c)=ab+ac$.

\textbf{Clases de ecuaciones}

\begin{itemize}
    \item Ecuaciones lineales: Ecuaciones en las que la variable tiene un exponente de 1.
    \item Ecuaciones cuadráticas (de segundo grado): Ecuaciones en las que la variable tiene un exponente de 2.
    \item Ecuaciones polinómicas: Ecuaciones que pueden tener variables con exponentes mayores que 2.
    \item Ecuaciones racionales: Ecuaciones que contienen fracciones con polinomios en el numerador y el denominador.
    \item Ecuaciones irracionales: Ecuaciones que contienen raíces de variables.
\end{itemize}


\subsubsection{Ecuaciones de primer grado}

Las ecuaciones de primer grado, o ecuaciones lineales, tienen la forma general $ax+b=0$, donde $a$ y $b$ son constantes y $a\neq0$.

\subsubsection{Ecuaciones de segundo grado}

Las ecuaciones de segundo grado, o ecuaciones cuadráticas, tienen la forma general $ax2+bx+c=0$, donde $a$, $b$ y $c$ son constantes y $a\neq0$.

Para resolver ecuaciones de segundo grado, se pueden usar varios métodos:
\begin{itemize}
    \item Factorización
    \item Completando el cuadrado
    \item Fórmula cuadrática
\end{itemize}

\subsection{Sistemas de ecuaciones}

Un sistema de ecuaciones es un conjunto de dos o más ecuaciones con las mismas variables. Resolver un sistema de ecuaciones significa encontrar los valores de las variables que satisfacen todas las ecuaciones simultáneamente.

\subsubsection{Sistemas de dos ecuaciones lineales con dos incógnitas}
\subsubsection{Sistemas de dos ecuaciones}

\subsection{Recta}
\subsubsection{Distancia entre dos puntos}
\subsubsection{Punto medio}
\subsubsection{Pendiente de una recta}
\subsubsection{Condiciones de paralelismo y perpendicularidad}

\subsection{Circunferencia}
\subsubsection{Circunferencia como lugar geométrico}
\subsubsection{Calcular la ecuación de una circunferencia con centro en el origen}
\subsubsection{Ecuación de la circunferencia forma general y forma canónica}
\subsubsection{Elementos de una circunferencia}

\subsection{Límites}
\subsubsection{Definición formal}
\subsubsection{Teoremas sobre límites}
\subsubsection{Evaluar límite}
\subsubsection{Límite indeterminado}

\subsection{La derivada}
\subsubsection{Definición de derivada}
\subsubsection{Derivar $x^3 - 3x^2 + x - 1$: Obtención de derivadas}
\subsubsection{Interpretación geométrica}
\subsubsection{Ecuación de la recta tangente y de la recta normal}
\subsubsection{Cálculo de la velocidad y aceleración de un móvil usando derivadas}
